\documentclass[UTF8]{ctexart}
\usepackage{geometry, CJKutf8}
\geometry{margin=1.5cm, vmargin={0pt,1cm}}
\setlength{\topmargin}{-1cm}
\setlength{\paperheight}{29.7cm}
\setlength{\textheight}{25.3cm}


% useful packages.
\usepackage{amsfonts}
\usepackage{amsmath}
\usepackage{amssymb}
\usepackage{amsthm}
\usepackage{enumerate}
\usepackage{graphicx}
\usepackage{multicol}
\usepackage{fancyhdr}
\usepackage{layout}
\usepackage{listings}
\usepackage{float, caption}

\lstset{
    basicstyle=\ttfamily, basewidth=0.5em
}

% some common command
\newcommand{\dif}{\mathrm{d}}
\newcommand{\avg}[1]{\left\langle #1 \right\rangle}
\newcommand{\difFrac}[2]{\frac{\dif #1}{\dif #2}}
\newcommand{\pdfFrac}[2]{\frac{\partial #1}{\partial #2}}
\newcommand{\OFL}{\mathrm{OFL}}
\newcommand{\UFL}{\mathrm{UFL}}
\newcommand{\fl}{\mathrm{fl}}
\newcommand{\op}{\odot}
\newcommand{\Eabs}{E_{\mathrm{abs}}}
\newcommand{\Erel}{E_{\mathrm{rel}}}

\begin{document}

\pagestyle{fancy}
\fancyhead{}
\lhead{郭子昊, 3230104714}
\chead{数据结构与算法第六次作业}
\rhead{Nov.10th, 2024}

\section{对修改后的remove函数的阐述和分析}
对于修改后的remove函数,首先检查节点是否为空。然后递归查找要删除的节点,小于当前元素就递归到左子树,反之就递归到右子树。
如果当前节点有两个子节点,那么通过detachMin函数找到右子树中的最小节点,替换当前节点并且删除。
如果当前节点只有一个子节点或者没有子节点,用非空子节点替换当前节点。
然后我用辅助函数balance来保持AVL树的平衡。
对于balance树,我的设计思路是先检查节点是否为空,然后检查并修正左子树的平衡,如果左子树高度大于右子树,那么就利用rotateWithLeftChild函数进行
单旋转,否则就利用doubleWithLeftChild进行双旋转,然后同理检查并且修正右子树,最后用updateHeight函数更新节点高度。


\section{测试的结果}

测试结果一切正常。详细的结果见上文的测试程序的设计思路部分。



\end{document}

%%% Local Variables: 
%%% mode: latex
%%% TeX-master: t
%%% End: 
