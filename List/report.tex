\documentclass[UTF8]{ctexart}
\usepackage{geometry, CJKutf8}
\geometry{margin=1.5cm, vmargin={0pt,1cm}}
\setlength{\topmargin}{-1cm}
\setlength{\paperheight}{29.7cm}
\setlength{\textheight}{25.3cm}


% useful packages.
\usepackage{amsfonts}
\usepackage{amsmath}
\usepackage{amssymb}
\usepackage{amsthm}
\usepackage{enumerate}
\usepackage{graphicx}
\usepackage{multicol}
\usepackage{fancyhdr}
\usepackage{layout}
\usepackage{listings}
\usepackage{float, caption}

\lstset{
    basicstyle=\ttfamily, basewidth=0.5em
}

% some common command
\newcommand{\dif}{\mathrm{d}}
\newcommand{\avg}[1]{\left\langle #1 \right\rangle}
\newcommand{\difFrac}[2]{\frac{\dif #1}{\dif #2}}
\newcommand{\pdfFrac}[2]{\frac{\partial #1}{\partial #2}}
\newcommand{\OFL}{\mathrm{OFL}}
\newcommand{\UFL}{\mathrm{UFL}}
\newcommand{\fl}{\mathrm{fl}}
\newcommand{\op}{\odot}
\newcommand{\Eabs}{E_{\mathrm{abs}}}
\newcommand{\Erel}{E_{\mathrm{rel}}}

\begin{document}

\pagestyle{fancy}
\fancyhead{}
\lhead{郭子昊, 3230104714}
\chead{数据结构与算法第四次作业}
\rhead{Oct.20th, 2024}

\section{测试程序的设计思路}

首先,我创建了一个空的List对象,以便后续进行各种操作的测试。

2,我通过pushfront和 pushback 的方法在链表的头部和尾部插入元素,并且打印链表中的内容来验证插入是否成功。我的思路是插入四个元素,分别用pushfront和pushback操作,观察四个元素的排列顺序即可证明。

根据“List elements after pushfront and pushback::5 10 20 25”的输出,发现插入成功了。

3,我通过size和empty方法来验证链表的大小和是否为空。因为上文我插入了四个元素,那么此时size的结果也应该是4,并且显然链表不是空链表,那么empty的结果应该是false。

根据“size: 4  Is empty: No”的输出,发现size和empty方法的结果是正确的。

4,我通过front和back方法来验证链表的头部和尾部元素是否正确。因为我插入了四个元素,那么头部元素应该是第一个插入的元素,尾部元素应该是最后一个插入的元素。

根据“Front: 5  Back: 25”的输出,发现front和back方法的结果是正确的。

(补充)我测试迭代器,将List1中的元素插入到迭代器中,观察迭代器的元素内容。

根据“List elements: 5 10 20 25 ”的输出,发现迭代器结果正确。

5,我通过popfront和popback方法来验证链表的头部和尾部元素是否被正确删除。因为我插入了四个元素,那么我popfront和popback之后,链表中应该只剩下两个元素。并且这两个元素应该是第二个和第三个插入的元素,即10和20。

根据“After popfront and popback, size: 2  List elements after popfront and popback: 10 20 ”的输出,发现 popfront 和 popback 方法的结果是正确的。

6,我通过insert的方法在链表插入元素,并且打印链表中的内容来验证插入是否成功。我的思路是在链表表头插入一个元素15,然后打印链表中的内容,观察此时链表的size并且观察15是否在表头。

根据“After insert, size: 3  List elements after insert: 15 10 20”的输出,发现insert方法的结果是正确的。

7,我通过erase的方法在链表删除元素,并且打印链表中的内容来验证删除是否成功。我的思路是删除链表中头元素,即刚才用insert插入的15,然后打印链表中的内容,观察此时链表的size并且观察15是否在链表中。

根据“After erase, size: 2  List elements after erase: 10 20”的输出,发现erase方法的结果是正确的。

8,我通过clear的方法清空链表,并且打印链表中的内容来验证清空是否成功。我的思路是清空链表,然后打印链表中的内容,观察此时链表的size是否为0。

根据“After clear, size: 0   ”的输出,发现clear方法的结果是正确的。

9,我测试初始化列表构造函数。使用初始化列表构造函数创建一个新的链表 list2,并初始化其内容为 {1, 2, 3, 4, 5},然后打印链表中的内容,通过观察链表的size和内容,来验证初始化是否成功。

根据“List2 size: 5  List2 elements: 1 2 3 4 5”的输出,发现初始化列表构造函数的结果是正确的。

10,我测试拷贝构造函数。使用拷贝构造函数创建一个新的链表 list3,并将 list2 拷贝给 list3,然后打印链表中的内容,通过观察链表的size和内容,来验证拷贝是否成功。

根据“List3 size (copy of list2): 5  List3 elements (copy of list2): 1 2 3 4 5 ”的输出,发现拷贝构造函数的结果是正确的。

11,我测试赋值运算符。我构建了一个新链表List4,然后将List3赋值给List4,然后打印链表中的内容,通过观察链表的size和内容,来验证赋值是否成功。

根据“List4 size (assigned from list3): 5  List4 elements (assigned from list3): 1 2 3 4 5 ”的输出,发现赋值运算符的结果是正确的。

12,我测试移动构造函数。我构建了一个新链表List5,然后将List4移动给List5,然后打印链表中的内容,通过观察链表5的size和内容,以及链表4的size是否为0,来验证移动是否成功。

根据“List5 size (moved from list4): 5  List4 size after move: 0  List5 elements (moved from list4): 1 2 3 4 5 ”的输出,发现移动构造函数的结果是正确的。

13,我测试前置自增和后置自增运算符。我取链表2中的头元素1,对它分别进行++iter,itre++,再iter++的操作,分别观察三次的输出情况。

根据“Initial value: 1  Value after ++iter: 2  Value after iter++: 2  Value after iter++: 3”的输出,发现前置自增和后置自增运算符的结果是正确的。

15,我测试前置自减和后置自减运算符。我取前文运算后得到的结果3,对它分别进行--iter,itre--,再iter--的操作,分别观察三次的输出情况。

根据“Initial value: 3  Value after --iter: 2  Value after iter--: 2  Value after iter--: 1”的输出,发现前置自减和后置自减运算符的结果是正确的。

16,对于迭代器的测试,我创建一个新链表List6,是5,10,15,20,25。让迭代器itr1先指向第一个元素5,然后将第一个元素修改成-5,输出两次结果后将迭代器指向
下一个元素10并输出。
然后我新建一个迭代器itr2,让itr2指向List6末端超出最后的一个元素的位置,此时应该输出结果是0,然后再令itr2指向前一个元素25,输出结果应该是25。
然后我让itr1和itr2分别前进后退一个位置,应该输出10和20,发现结果正确。
最后分别测试与自身的相等,他元素相等,自身不等,他元素不等,判断1和0的true or false。

\section{测试的结果}

测试结果一切正常。详细的结果见上文的测试程序的设计思路部分。

我用 valgrind 进行测试,发现没有发生内存泄露。

\section{(可选)bug报告}

我发现了一个 bug,触发条件如下:

\begin{enumerate}
    \item 首先……
    \item 然后……
    \item 此时发现……
\end{enumerate}

据我分析,它出现的原因是:

\end{document}

%%% Local Variables: 
%%% mode: latex
%%% TeX-master: t
%%% End: 
