\documentclass[UTF8, a4paper, 12pt]{ctexart}

%=================== 宏包引用 ===================
\usepackage{geometry}           % 页面布局
\usepackage{amsfonts, amsmath, amssymb, amsthm} % 数学公式符号
\usepackage{graphicx}           % 图片插入
\usepackage{float}              % 浮动体控制
\usepackage{caption}            % 图表标题
\usepackage{fancyhdr}           % 页眉页脚
\usepackage{listings}           % 代码抄录
\usepackage{xcolor}             % 颜色支持
\usepackage{booktabs}           % 三线表
\usepackage{hyperref}           % 超链接与书签
% 在导言区添加 TikZ 支持
\usepackage{tikz}
\usetikzlibrary{shapes, arrows, positioning, calc}

%=================== 页面设置 ===================
\geometry{left=2.5cm, right=2.5cm, top=2.5cm, bottom=2.5cm}

%=================== 标题格式修正 ===================
% 强制一级标题左对齐,取消默认的居中
\ctexset{
    section = {
        format = \Large\bfseries\raggedright, % 左对齐
        aftername = \hspace{1em} 
    }
}

%=================== 代码高亮设置 ===================
\definecolor{codegreen}{rgb}{0,0.6,0}
\definecolor{codegray}{rgb}{0.5,0.5,0.5}
\definecolor{codepurple}{rgb}{0.58,0,0.82}
\definecolor{backcolour}{rgb}{0.95,0.95,0.92}

\lstset{
    language=Matlab,
    backgroundcolor=\color{backcolour},   
    commentstyle=\color{codegreen},
    keywordstyle=\color{blue},
    numberstyle=\tiny\color{codegray},
    stringstyle=\color{codepurple},
    basicstyle=\ttfamily\footnotesize,
    breakatwhitespace=false,         
    breaklines=true,                 
    captionpos=b,                    
    keepspaces=true,                 
    numbers=left,                    
    numbersep=5pt,                  
    showspaces=false,                
    showstringspaces=false,
    showtabs=false,                  
    tabsize=2,
    frame=single
}

%=================== 常用数学命令 ===================
\newcommand{\dif}{\mathrm{d}}
\newcommand{\avg}[1]{\left\langle #1 \right\rangle}
\newcommand{\difFrac}[2]{\frac{\dif #1}{\dif #2}}
\newcommand{\pdfFrac}[2]{\frac{\partial #1}{\partial #2}}

%=================== 页眉页脚设置 ===================
\pagestyle{fancy}
\fancyhead{} 
\lhead{郭子昊, 3230104714}
\chead{智能控制技术:神经网络控制作业}
\rhead{\today}
\cfoot{\thepage}

%=================== 文档开始 ===================
\begin{document}

%------------------- 封面部分 -------------------
\begin{center}
    \vspace*{1cm}
    \huge\bfseries 智能控制技术作业报告\\[0.5cm]
    \Large —— 基于神经网络的 USV 航向控制\\[2cm]
    
    \large
    \begin{tabular}{rl}
        \textbf{姓名:} & 郭子昊 \\
        \textbf{学号:} & 3230104714 \\
        \textbf{课程:} & 智能控制技术 \\
        \textbf{日期:} & \today \\
    \end{tabular}
\end{center}
\vspace{2cm}
\thispagestyle{empty}
\newpage
\setcounter{page}{1}

%------------------- 正文部分 -------------------

\section{问题描述与被控对象}

\subsection{问题描述}

考虑与前面作业中相同的无人水面艇(USV)航向跟踪控制系统,通过调节舵角 $\delta$,使船舶航向 $\psi$ 能够跟踪时变的参考航向 $\psi_{ref}(t)$,参考信号为正弦信号:
\begin{equation}
    \psi_{ref}(t) = 0.2 \sin(0.1t)
\end{equation}

具体实验任务如下:
\begin{enumerate}
    \item 在 Matlab 中利用已给的带扰动噪声系统模型采集系统的输入输出数据,采用 BP 神经网络作为辨识模型,建立 USV 的神经网络辨识模型,并分析辨识结果的精度。
    \item 设计一个 BP 神经网络直接作为控制器,画出该直接神经网络控制系统的结构框图,利用梯度下降法,推导该神经网络控制器的权值在线更新律,并进行控制系统仿真。在仿真中分别采用符号近似法和神经网络估计法估计未知系统 Jacobian。
    \item 用一个 BP 神经网络在线调整 PID 控制器的参数,画出该神经网络 PID 控制系统的结构框图,同样利用梯度下降法,推导神经网络的权值更新律,并进行控制系统仿真。
    \item (附加题)借鉴模糊自适应相同的方式,设计针对仿射非线性 USV 模型的神经网络自适应控制器,给出控制律和自适应律,并进行仿真。(提示:模糊逼近可以用 RBF 神经网络逼近代替,RBF 网络即为激发函数为高斯函数的前向网络)。
\end{enumerate}

\subsection{USV 数学模型}
无人艇的非线性动力学模型如下:
\begin{equation}
    T\dot{\psi} + K H(\dot{\psi}) = K \delta + d(t)
    \label{eq:usv_model}
\end{equation}
其中,非线性函数 $H(\dot{\psi}) = \alpha \dot{\psi} + \beta \dot{\psi}^3$。相关参数设置为 $T=2.0s, K=0.8, \alpha=1.0, \beta=0.5$。

\begin{figure}[H]
    \centering
    \includegraphics[width=0.8\textwidth]{figure/USV_model.png}
    \caption{USV模型示意图}
    \label{fig:usv_model}
\end{figure}

注意到,在此前的作业中,我已经详细地分析和实现了该 USV 模型的 Simulink 仿真模块,此处不再赘述,直接利用该模型进行神经网络控制实验。
% 将 & 改为 and 以避免书签报错
\section{任务一:神经网络系统辨识}
整体神经网络的离线辨识建模如下:
\begin{figure}[H]
    \centering
    \includegraphics[width=0.8\textwidth]{figure/nn_identification.png}
    \caption{神经网络辨识系统结构图}
    %\label{fig:nn_identification}
\end{figure}

\subsection{设计思路与数据采集}

设计思路如下,
进行控制系统设计之前,首要任务是建立被控对象(USV)的精确数学模型或辨识模型。尽管已知 USV 的动力学方程,但在实际工程中,非线性项(如 $H(\dot{\psi})$)往往难以精确获得。
这是工程意义上的原因,客观意义上,结合课堂上的学习,我采用 BP 神经网络作为黑箱辨识器,利用输入输出数据来逼近 USV 的非线性动态特性。
USV 是连续系统,而神经网络是离散计算的。设定采样时间 $T_s=0.1s$,采集离散的输入序列 $u(k)$ 和输出序列 $y(k)$。

在数据采集实现上,我在 Simulink 中搭建了数据采集回路。模型如下,对应文件夹中的untitled.slx文件。
\begin{figure}[H]
    \centering
    \includegraphics[width=0.8\textwidth]{figure/data_collect.png}
    \caption{数据采集回路}
    \label{fig:nn_identification}
\end{figure}

我通过功率为0.0001的随机白噪声叠加一个正弦波模块产生一组随机变化的舵角信号 $\delta$ 输入给 USV 模型,
并记录对应的航向角 $\psi$。在采样时间为2000s,步长为0.1s下,共采集了 20000 组样本数据,分为训练集(前15000组)和测试集(后5000组),用于神经网络的离线预训练。

\subsection{神经网络辨识模型搭建}

基于采集的数据,我设计了一个基于 BP 算法的前馈神经网络辨识器(NN Identifier)。
\subsubsection{网络结构设计}
根据 USV 的二阶系统特性,为了完整捕获系统动态,选取了 5 个变量作为神经网络的输入层节点:
\begin{equation}
    x(k) = [y(k), y(k-1), y(k-2), u(k), u(k-1)]^T
\end{equation}
其中,$y(k)$ 为当前时刻的航向角,$u(k)$ 为当前时刻的舵角。

网络结构参数设置如下:
\begin{itemize}
    \item 输入层:5 个神经元。
    \item 隐含层:30 个神经元。节点数根据经验公式及反复测试确定,采用 $\tanh$ (双曲正切) 作为激活函数,以捕捉非线性特征。
    \item 输出层:1 个神经元。对应下一时刻的预测输出 $\hat{y}(k+1)$,采用 Linear (线性) 激活函数。
\end{itemize}

\subsubsection{训练过程实现}
我编写了 MATLAB 脚本对采集到的输入输出数据进行处理与离线训练,实际代码如下,保存为文件train\_usv.m在文件夹中。

\begin{lstlisting}
%% 1. 数据提取
if exist('out', 'var')
    u_data = out.sim_u;
    y_data = out.sim_y;
else
    error('错误:未找到 out 变量!请先运行 Simulink 模型采集数据。');
end

% 数据长度
L = length(u_data);

%% 2. 构建训练集 

start_idx = 3; 
end_idx = L - 1;

P_train = [y_data(start_idx:end_idx)'; ...     % y(k)
           y_data(start_idx-1:end_idx-1)'; ... % y(k-1)
           y_data(start_idx-2:end_idx-2)'; ... % y(k-2) 
           u_data(start_idx:end_idx)'; ...     % u(k)
           u_data(start_idx-1:end_idx-1)'];    % u(k-1) 

T_train = y_data(start_idx+1:end_idx+1)';      % y(k+1)

%% 3. 训练
net = feedforwardnet(30, 'trainlm');
net.trainParam.goal = 1e-12;          % 目标设高一点
net.trainParam.min_grad = 1e-15;   % 防止因梯度太小提前停止
net.trainParam.epochs = 1000;
[net, tr] = train(net, P_train, T_train);

%% 4. 生成模块
gensim(net, -1);

disp('已生成 Simulink 神经网络模块,请查看弹出的窗口。');
\end{lstlisting}

训练后得到的神经网络模型在 Simulink 中生成了一个可调用的模块,命名为 Feed-Forward Neural Network。
图片如下所示,相关模型亦保存为untitled1.slx文件在文件夹中。
\begin{figure}[H]
    \centering
    \includegraphics[width=0.5\textwidth]{figure/lsmodel.png}
    \caption{离线神经网络模块示意图}
    \label{fig:lsmodel}
\end{figure}

\subsection{辨识结果分析}
基于搭建的神经网络辨识模型,在随机舵角信号激励下进行了在线辨识仿真验证,结果如下图所示。

\begin{figure}[H]
    \centering
    \includegraphics[width=0.9\textwidth]{figure/TASK1.png}
    \caption{神经网络辨识性能验证:(左) 辨识输出 $\hat{y}$ 与真实输出 $y$ 对比;(右) 辨识误差演化曲线}
    \label{fig:TASK1}
\end{figure}

从仿真结果可以得出以下分析结论:

\begin{enumerate}
    \item 动态逼近能力强:观察左图可知,神经网络的预测输出 $\hat{y}(k)$(蓝线)与 USV 的真实航向响应 $y(k)$(黄线)几乎完全重合。无论是在信号的上升沿、下降沿还是峰值处,网络均未表现出明显的相位滞后或幅值衰减,说明该模型具备足够的模型容量来捕捉 USV 的二阶非线性惯性特征。  
    \item 辨识精度满足要求:右图展示了辨识误差 $e(k) = y(k) - \hat{y}(k)$ 的时域变化。可以看出,在仿真初期的短暂权值调整后,误差迅速收敛并稳定在 $\pm 0.02$ rad 的极小范围内(约为信号幅值的 1\% 以下)。这种低数量级的残差表明,神经网络已经成功建立了从舵角输入到航向输出的高精度映射。   
\end{enumerate}

综上所述,该神经网络辨识器性能优异,能够为后续的控制器设计提供可靠的模型基础。

\section{任务二:基于 BP 神经网络的直接控制}

\subsection{控制系统结构设计}

在本环节中,我需要设计一个 BP 神经网络直接充当控制器,神经网络控制器通过学习从误差状态到控制量 $u$ 的映射关系,直接输出舵角 $\delta$。

\subsubsection{系统结构框图}
该直接控制系统的结构如图所示。系统主要由三个部分组成:
\begin{enumerate}
    \item 控制器:一个三层 BP 神经网络,输入为参考航向 $y_r$ 与实际航向 $y$ 的误差信号,输出为控制量 $u(k)$。
    \item 被控对象:USV 非线性模型。
    \item 辨识器:即神经网络,用于在“神经网络估计法”中提供系统的灵敏度信息(Jacobian)。
\end{enumerate}

\begin{figure}[H]
    \centering
    
    \begin{tikzpicture}[auto, node distance=2cm,>=latex',
        % 定义样式
        % 蓝色方框样式 (控制器和对象)
        block/.style = {draw, rectangle, minimum height=1.2cm, minimum width=3.0cm, align=center, fill=blue!5}, 
        % 黄色圆角方框样式 (估计器) - 稍微加宽以适应中文 multiline text
        est/.style = {draw, rectangle, minimum height=1.4cm, minimum width=4.2cm, align=center, fill=yellow!10, rounded corners}, 
        sum/.style = {draw, circle, inner sep=2pt, node distance=1.5cm},
        input/.style = {coordinate},
        output/.style = {coordinate}
    ]

        % 1. 定义节点 (Nodes)
        \node [input] (input) {};
        \node [sum, right of=input] (sum) {};
        
        % 翻译: BP Neural Controller -> BP 神经网络控制器
        \node [block, right of=sum, node distance=3.8cm] (ctrl) {\textbf{BP 神经网络}\\\textbf{控制器}};
        
        % 翻译: Plant (USV) -> 被控对象 (USV)
        \node [block, right of=ctrl, node distance=4.2cm] (plant) {\textbf{被控对象}\\\textbf{(USV)}};
        \node [output, right of=plant, node distance=2.5cm] (out) {};
        
        % Jacobian 估计器 (放在下方)
        % 翻译标题: Jacobian Estimator -> Jacobian 估计器
        % 翻译子标题: (NN Identifier / Sign) -> (神经网络辨识器 / 符号函数)
        \node [est, below of=ctrl, node distance=2.8cm, xshift=2.2cm] (est) {\textbf{Jacobian 估计器}\\(\small 神经网络辨识器 / 符号函数)};

        % 2. 绘制连线 (Connections) - 前向通路 (黑色实线)
        \draw [->, thick] (input) -- node {$y_r$} (sum);
        \draw [->, thick] (sum) -- node {$e$} (ctrl);
        \draw [->, thick] (ctrl) -- node [name=u] {$u$} (plant);
        \draw [->, thick] (plant) -- node [name=y] {$y$} (out);
        
        % 比较器的符号 (+/-)
        \node at (sum.north west) [xshift=-2pt, yshift=2pt] {$+$};
        \node at (sum.south) [yshift=-5pt, xshift=-5pt] {$-$};

        % 3. 绘制反馈回路 (Feedback)
        % 反馈线适当调低以避免穿过估计器模块
        \draw [->, thick] ($(plant.east)+(1.5cm,0)$) |- (0,-4.5) -| (sum);
        
        % 4. 绘制 Jacobian 信息通路 (红色虚线 - 学习通道)
        % u -> Estimator 分叉点
        \coordinate (u_split) at ($(ctrl.east)!0.5!(plant.west)$);
        \draw [dashed] (u_split) |- (est.east); 
        
        % Estimator -> Controller (梯度信息反馈)
        % 翻译标签: Jacobian Info \partial y / \partial u -> Jacobian 信息 \partial y / \partial u
        \draw [->, dashed, red, line width=1pt] (est.west) -| node [pos=0.2, left, text=red, align=center] {Jacobian 信息\\$\partial y / \partial u$} (ctrl.south);

    \end{tikzpicture}
    \caption{ BP 神经网络直接控制系统结构框图}
    \label{fig:direct_control_structure_cn}

\end{figure}

\subsection{控制器权值更新律推导}

\subsubsection{性能指标函数}
定义时刻 $k$ 的二次型性能指标函数 $J(k)$ 为:
\begin{equation}
    E(k) = \frac{1}{2} \left( y_d(k+1) - y(k+1) \right)^2 = \frac{1}{2} e(k+1)^2
\end{equation}
其中,$y_d$ 为期望输出,$y$ 为实际输出。

\subsubsection{梯度下降法推导}
根据梯度下降法,控制器权值 $w$ 的调整量 $\Delta w$ 应沿着性能指标函数梯度的负方向进行:
\begin{equation}
    \Delta w(k) = -\eta \frac{\partial E(k)}{\partial w(k)}
\end{equation}
其中 $\eta$ 为学习率。

利用链式法则展开梯度项:
\begin{equation}
    \frac{\partial E(k)}{\partial w(k)} = \frac{\partial E(k)}{\partial y(k+1)} \cdot \frac{\partial y(k+1)}{\partial u(k)} \cdot \frac{\partial u(k)}{\partial w(k)}
\end{equation}

代入误差定义 $\frac{\partial E}{\partial y} = -(y_d - y) = -e$,则有:
\begin{equation}
    \frac{\partial E(k)}{\partial w(k)} = -e(k+1) \cdot \underbrace{\frac{\partial y(k+1)}{\partial u(k)}}_{\text{Jacobian Info}} \cdot \underbrace{\frac{\partial u(k)}{\partial w(k)}}_{\text{NN Gradient}}
    \label{eq:chain_rule}
\end{equation}

由此得到权值在线更新律:
\begin{equation}
    w(k+1) = w(k) + \eta \cdot e(k+1) \cdot \frac{\partial y(k+1)}{\partial u(k)} \cdot \frac{\partial u(k)}{\partial w(k)}
    \label{eq:update_law}
\end{equation}

其中:
\begin{itemize}
    \item $e(k+1)$:可直接观测。
    \item $\frac{\partial u(k)}{\partial w(k)}$:可通过 BP 神经网络的反向传播算法计算。
    \item $\frac{\partial y(k+1)}{\partial u(k)}$:即被控对象的Jacobian 信息(系统灵敏度)。由于 USV 模型是非线性的且环境未知,该项无法直接获得,需要进行估计。
\end{itemize}

需要注意的是,该项是在施加控制量 $u(k)$ 并经过一个采样周期 $T$ 后,在 $k+1$ 时刻实际测量得到的。
这意味着权值的更新属于事后修正环节:即利用 $k+1$ 时刻观测到的结果,来修正网络参数以优化下一时刻 $k+1$ 的控制输出。

在上述权值更新律中,$\frac{\partial u(k)}{\partial w(k)}$ 项反映了神经网络内部参数变化对控制输出的影响。
本实验的直接神经控制器采用 2-6-1 结构,其中隐含层选用 Sigmoid 函数以引入非线性,输出层选用线性函数以覆盖舵角的全动态范围。具体推导如下:

\paragraph{1. 输出层权值梯度 (线性激活)}
设输出层输入为 $net^{(2)}$,输出为 $u(k)$。由于输出层采用线性激活函数(Linear Activation):
\begin{equation}
    u(k) = net^{(2)} = \sum_{j=1}^{m} w_{j}^{(2)} H_j
\end{equation}
其中 $H_j$ 为隐层第 $j$ 个神经元的输出。因此,控制量 $u$ 对输出层权值 $w_{j}^{(2)}$ 的偏导数为:
\begin{equation}
    \frac{\partial u(k)}{\partial w_{j}^{(2)}} = H_j
\end{equation}
这对应了代码中输出层权值的更新逻辑:\texttt{dW2 = xite * delta3 * H'}。

\paragraph{2. 隐含层权值梯度 (Sigmoid 激活)}
隐含层神经元采用 Sigmoid 函数:
\begin{equation}
    H_j = \sigma(net_j^{(1)}) = \frac{1}{1 + e^{-net_j^{(1)}}}
\end{equation}
利用 Sigmoid 导数性质 $\sigma'(x) = \sigma(x)(1-\sigma(x))$,隐含层输出对输入的灵敏度为:
\begin{equation}
    \frac{\partial H_j}{\partial net_j^{(1)}} = H_j (1 - H_j)
\end{equation}

\paragraph{3. 完整的权值更新算式}
综合上述推导,结合 Jacobian 信息 $\text{sgn}(\frac{\partial y}{\partial u})$,可得具体的权值修正公式:

对于输出层权值 $w^{(2)}$:
\begin{equation}
    \Delta w^{(2)}(k) = \eta \cdot e(k+1) \cdot \text{sgn}\left(\frac{\partial y}{\partial u}\right) \cdot H^T
\end{equation}

对于隐层权值 $w^{(1)}$(利用反向传播将误差传递至隐层):
\begin{equation}
    \Delta w^{(1)}(k) = \eta \cdot \delta^{(1)} \cdot x^T
\end{equation}
其中隐层局部梯度 $\delta^{(1)}$ 为:
\begin{equation}
    \delta^{(1)} = \left[ (w^{(2)})^T \cdot \left( e(k+1) \cdot \text{sgn}\left(\frac{\partial y}{\partial u}\right) \right) \right] \cdot \underbrace{H(1-H)}_{\text{激活函数导数}}
\end{equation}
以上推导完整揭示了直接神经网络控制器的权值在线更新机制,关键在于如何有效估计未知的 Jacobian 信息 $\frac{\partial y}{\partial u}$,这将在下一节中详细讨论。

\subsection{Jacobian 信息的两种估计方法}

为了解决未知 Jacobian 项 $\frac{\partial y}{\partial u}$ 的获取问题,我通过课程的学习,分别采用了以下两种方法进行对比仿真。

\subsubsection{方法一:符号近似法}
该方法基于对被控对象物理特性的定性分析。对于 USV 航向控制,舵角 $\delta$ 的变化方向与航向 $\psi$ 的变化方向通常是单调对应的,
通过简单的开环测试,我得到了符号的正确结论,也就是正舵角导致正向转首。
因此,可以用梯度的符号函数 $\text{sgn}(\frac{\partial y}{\partial u})$ 来代替精确值,并将幅值信息归并到学习率 $\eta$ 中。

此时,Jacobian 项近似为:
\begin{equation}
    \frac{\partial y(k+1)}{\partial u(k)} \approx \text{sgn} \left( \frac{\partial y}{\partial u} \right)
\end{equation}

在实际代码实现中,我将该符号值设为常数 +1,进行后续的计算。
这种方法计算简单,不需要额外的辨识模型,但由于缺乏精确的梯度幅值信息,可能导致收敛速度较慢或在平衡点附近产生震荡。

符号似然法的模型实现如下图:
\begin{figure}[H]
    \centering
    \includegraphics[width=0.8\textwidth]{figure/sign_approximation.png}
    \caption{符号近似法的神经网络直接控制系统结构图}
    \label{fig:sign_approximation}
\end{figure}

接下来我将附上我的符号近似法的神经网络控制器核心代码,如下所示:
\begin{lstlisting}
function [u, W1_out, W2_out] = fcn(yr, y)

persistent W1 W2 e_prev is_init_final_trained

% --- 1. 参数设置 ---
xite  = 0.001;   
alpha = 0.05;  

% --- 2. 初始化---
if isempty(is_init_final_trained)

    W1 = [  2.8620, -0.04419;
            0.8460, -0.009854;
           -1.1170,  0.04621;
           -1.0690,  0.04619;
           -1.0230, -0.03447;
           -0.9713,  0.04703 ]; 
           
    W2 = [ 3.853, 0.646, -1.196, -1.149, -1.102, -1.05 ];
    
    e_prev = 0;
    is_init_final_trained = 1; 
end

% --- 3. 输入处理 ---
e = yr - y;
e_delta = e - e_prev; 
xi = [e; e_delta]; 

% --- 4. 前向计算 ---
net2 = W1 * xi;
H = 1 ./ (1 + exp(-net2)); 
net3 = W2 * H;
u_raw = net3; 

% 物理限位
if u_raw > 0.5
    u = 0.5;
elseif u_raw < -0.5
    u = -0.5;
else
    u = u_raw;
end

% --- 5. 反向传播---
sgn_Jacobian = 1; 

% 权重更新
delta3 = e * sgn_Jacobian * 1;

% 输出层
dW2 = xite * delta3 * H';
W2 = W2 + dW2 + alpha * dW2;

% 隐层
d_sigma = H .* (1 - H) + 0.05; 
delta2 = (W2' * delta3) .* d_sigma;

dW1 = xite * delta2 * xi';
W1 = W1 + dW1 + alpha * dW1;

% --- 6. 更新历史 ---
e_prev = e;

W1_out = W1;
W2_out = W2;

end
\end{lstlisting}
\paragraph{代码详解与工程思考:}
在控制调节的初期,我的初始化代码一开始采用的随机初始化权值的方法。然而,在实际仿真中发现,这种随机初始化往往导致控制器在初始阶段表现出剧烈的震荡,甚至无法稳定跟踪参考信号。
后来经过我的调节和反复测试,我在一组训练的结果出发现神经网络在后期的控制效果较好,而前期因为初始化权值的随机和权值向正确方向不断调整,
导致初期控制效果不佳。具体实验情况如下图:
\begin{figure}[H]
    \centering
    \includegraphics[width=0.45\textwidth]{figure/direct_control_bad_init.png}
    \caption{随机初始化权值导致的初期控制震荡}
    \label{fig:bad_init}
\end{figure}

于是我开始思考猜想,如果我将后期训练好的权值直接加载进来,是否能避免初期的剧烈震荡,从而让控制器一开始就具备一定的稳态保持能力。
所以我通过display输出最终的权值,并将其硬编码进初始化代码中,效果显著提升。具体代码如下:
\begin{lstlisting}
if isempty(is_init_final_trained)
    W1 = [ 2.8620, -0.04419; ... ]; % 预训练好的权值
    W2 = [ 3.853, 0.646, ... ];
    ...
end
\end{lstlisting}

在计算网络输出后,我引入了物理限幅环节,以模拟真实舵机的机械约束:
\begin{lstlisting}
% 物理限位
if u_raw > 0.5
    u = 0.5;
elseif u_raw < -0.5
    u = -0.5;
...
\end{lstlisting}

这一设计防止了控制器输出超出执行机构的物理极限($\pm 0.5$ rad),避免了执行器饱和带来的积分发散风险。

在反向传播部分,最关键的创新点在于对 Jacobian 信息的简化处理:
\begin{lstlisting}
sgn_Jacobian = 1; 
delta3 = e * sgn_Jacobian * 1;
\end{lstlisting}

我将未知的 Jacobian 信息 $\frac{\partial y}{\partial u}$ 简化为常数 \texttt{sgn\_Jacobian = 1}。
这意味着控制器通过定性判断来决策:只要存在正向误差,就沿着增加 $u$ 的方向更新权值。这种策略虽然牺牲了部分动态响应的精细度,但有效地规避了复杂的在线辨识过程,显著提高了算法的实时性。

同时,在反向传播更新权值时,我对标准的 Sigmoid 导数进行了改进:
\begin{lstlisting}
% 隐层梯度计算
d_sigma = H .* (1 - H) + 0.05; 
delta2 = (W2' * delta3) .* d_sigma;
\end{lstlisting}

标准的 Sigmoid 导数为 $f'(x) = f(x)(1-f(x))$。当神经元输出接近 0 或 1 ,也就是进入饱和区时,导数趋近于 0,会导致梯度消失,使得权值停止更新,系统陷入“死区”。
为此,我在代码中人为增加了一个偏置项 \texttt{+ 0.05}。这一微小的改进保证了即使神经元进入饱和状态,梯度依然保持非零,权值仍能获得持续的更新动力。

最后,在权值更新时,我引入了动量项以加速收敛:
\begin{lstlisting}
W1 = W1 + dW1 + alpha * dW1;
\end{lstlisting}

最终符号似然法的神经网络直接控制器的仿真结果如下图所示:
\begin{figure}[H]
    \centering
    \includegraphics[width=0.9\textwidth]{figure/direct_control_sign.png}
    \caption{符号似然法的神经网络直接控制系统仿真结果}
    \label{fig:direct_sign_res}
\end{figure}

不难发现,系统能够实现基本的跟踪功能,误差精度较好,仅为$10^{-3}$数量级,较出色地完成了控制任务。
相较而言,波峰波谷处存在略大一些的超调和稳态误差。这是因为固定的符号梯度无法适应 USV 在不同航速和流场下的动态增益变化。

\subsubsection{方法二:神经网络估计法}
与符号近似法仅利用定性结论不同,神经网络估计法充分利用神经网络辨识器。
该方法的核心思想是:利用辨识器所蕴含的数学模型,实时计算被控对象对控制输入的灵敏度信息(Jacobian)。


根据的辨识结果,我们有 $\hat{y}(k+1) \approx y(k+1)$。因此,可以通过计算辨识器网络的偏导数来逼近真实的系统梯度:
\begin{equation}
    \frac{\partial y(k+1)}{\partial u(k)} \approx \frac{\partial \hat{y}(k+1)}{\partial u(k)}
\end{equation}

该偏导数可以通过对辨识器神经网络进行反向传播精确计算得到。这种方法不再依赖固定的符号常数,而是能够根据当前的系统状态(航向、航速、流场干扰等),
动态地输出包含正确方向和精确幅值的梯度信息,从而指导控制器进行更精细的权值调整。

神经网络估计法的控制系统实现如下图所示:
\begin{figure}[H]
    \centering
    \includegraphics[width=0.8\textwidth]{figure/direct_control_NN.png}
    \caption{神经网络估计法的神经网络直接控制系统结构图}
    \label{fig:nn_estimation_structure}
\end{figure}


我将整个神经网络估计法的模型分成了两个部分,分别是“控制器模块”和“辨识器模块”,也就是图片中的NN\_Controller和NN\_Identifier。
其中,辨识器模块采取在线辨识的代码实现,具体代码如下所示:

\begin{lstlisting}
function [J_est, y_hat] = fcn_Identifier(u, y)

persistent W1 W2 b1 b2 u_prev u_prev2 y_prev y_prev2 is_init

% --- 参数设置 ---
xite_i = 0.05;   
alpha_i = 0.05;

if isempty(is_init)
    
    % W1 (30行, 5列) 
    W1 = [ 1.1550    0.7454    0.9150   -1.5408   -1.6229;
           1.2152   -1.3472   -0.6444   -1.1061    1.2901;
          -1.5373   -0.9018    0.0650    0.3487   -1.7547;
           1.7296   -1.6454    0.5698   -0.3121    0.4781;
           0.5190   -1.3520    1.4344    1.3670    0.8669;
          -1.2576    0.9733    1.3336   -0.8581    1.1981;
          -1.1308    1.0980    0.3079    0.4444   -2.1551;
           0.5900   -0.9220   -2.0974   -0.0998   -0.6876;
           1.0608    1.0588   -1.3490   -1.3189   -0.2222;
           1.5665   -1.4511   -0.5776   -0.4067    0.4901;
          -1.5240   -0.3283    1.4332   -1.0913   -0.1751;
           1.7472   -0.4401   -0.9080    1.1144    1.4263;
           1.7463    1.0072    1.1938   -0.7705   -1.2222;
          -0.3095    1.5289   -1.8743    0.4432   -0.6706;
           1.0400   -1.1264    1.5059   -1.1677   -0.8901;
          -1.2790    0.4603   -0.2956    0.9907   -1.3388;
          -0.3413   -0.2105   -1.4706   -1.3012    1.5064;
           1.2787    0.0876   -1.1741   -0.0728    0.0747;
           1.9382    1.4027    0.7991    1.0716    0.1406;
           1.7422    0.9102   -0.2330    0.8941   -1.3070;
           1.3368   -1.0114   -0.7812   -0.4313    0.5677;
          -1.8533    0.4552    0.9058   -1.4618    0.2583;
           1.9096    0.8322    0.4428   -1.1921   -0.8640;
           1.8041   -1.4068   -0.0001    1.3022   -0.3553;
           0.4213   -1.7609    1.4647   -1.0283   -0.4767;
           1.2781    0.1658   -0.7475    1.4139   -1.6666;
           1.1649    2.1033    1.2228    0.1538   -1.1324;
          -0.3621   -0.5653    1.0707    1.8713   -1.4119;
           0.7797    0.4307   -0.5723   -1.8847   -1.2332;
          -1.7767   -1.4851    0.3863   -0.2859   -1.5525 ]; 
    
    % b1 (30行, 1列) 
    b1 = [-2.6178; -2.5214;  1.9676; -1.7376; -2.2797;
           1.6096;  1.6080; -1.3758; -1.3612; -0.3427;
           0.5525; -0.4758; -0.4769;  0.2875; -0.3108;
          -0.5670;  0.0213;  0.1445;  0.6710;  1.0964;
           1.0013; -1.4389;  1.2160;  1.2777;  1.7008;
           2.3905;  1.6822; -2.5693;  2.5941; -2.4270 ];

    % W2 (1行, 30列)
    W2 = [ -0.0002, -0.8092, -0.0001,  0.6664, -0.0068,  0.1702, -0.0026, -0.0008, -0.0007, ...
           -0.6598,  0.0105,  0.0000,  0.0000, -0.3008, -0.0000, -0.0437,  0.0024,  1.7531, ...
            0.0003, -0.0031, -0.5625, -0.0489,  0.0001, -0.0809,  0.0014, -0.3007,  0.0003, ...
            0.0603, -0.0001, -0.0007 ];
    
    % b2 (标量)
    b2 = 0.0617;
    
    % 初始化历史状态
    u_prev = 0; u_prev2 = 0;
    y_prev = 0; y_prev2 = 0;
    is_init = 1;
end

%  PART 1: 构造输入与前向计算
xi_i = [y; y_prev; y_prev2; u; u_prev];

% 前向计算 (使用 tanh)
net_1 = W1 * xi_i + b1;      
Hi = tanh(net_1);            
y_hat = W2 * Hi + b2;        

%  PART 2: 在线自适应 (反向传播)
e_id = y - y_hat; 

% 更新输出层
delta3 = e_id; 
dW2 = xite_i * delta3 * Hi';
db2 = xite_i * delta3;

W2 = W2 + dW2 + alpha_i * dW2;
b2 = b2 + db2 + alpha_i * db2;

% 更新隐层
d_sigma = 1 - Hi.^2; % tanh 导数
delta2 = (W2' * delta3) .* d_sigma;

dW1 = xite_i * delta2 * xi_i';
db1 = xite_i * delta2;

W1 = W1 + dW1 + alpha_i * dW1;
b1 = b1 + db1 + alpha_i * db1;

%  PART 3: 计算 Jacobian 并输出
raw_J = W2 * (d_sigma .* W1(:, 4)); 

J_est = abs(raw_J); % 物理约束

% 更新历史状态
y_prev2 = y_prev;
y_prev = y;

u_prev2 = u_prev;
u_prev = u;

end
\end{lstlisting}

我在书写代码时,特别注意了辨识器的初始化处,我大胆猜想辨识模块的初始化权值也可以进行预处理权值输入,也就是
将任务一中训练好的权值直接加载进来,从而提升辨识器的初始性能,结果表明这一方法同样有效,辨识器能够更快地收敛到较低的误差水平。同时
还具有在线自适应能力,能够持续优化辨识精度。结果如下图所示:
\begin{figure}[H]
    \centering
    \includegraphics[width=0.9\textwidth]{figure/nn_identifier_performance.png}
    \caption{神经网络辨识器的在线自适应性能}
    \label{fig:nn_identifier_performance}
\end{figure}

发现在经过前期一段时间的在线自适应后,辨识误差迅速降低并稳定在$2\times10^{-3}$的极小范围内,表现出优异的辨识能力。

接下来我将附上我的神经网络估计法的神经网络控制器核心代码:
\begin{lstlisting}
function u = fcn_Controller(yr, y, J_est_in)

persistent Wc1 Wc2 y_prev is_init

% --- 参数设置 ---
xite_c = 0.0001; 
alpha_c = 0.00;

if isempty(is_init)
    Wc1 = [  2.8620, -0.04419;
             0.8460, -0.009854;
            -1.1170,  0.04621;
            -1.0690,  0.04619;
            -1.0230, -0.03447;
            -0.9713,  0.04703 ]; 
           
    Wc2 = [ 3.853, 0.646, -1.196, -1.149, -1.102, -1.05 ];
    
    y_prev = 0;
    is_init = 1;
end

% 1. 安全底线 (处理 J_est)
if J_est_in < 0.001
    J_active = 0.001;
else
    J_active = J_est_in;
end

% 2. 构造输入
e = yr - y;
e_delta = e - (yr - y_prev);
xi_c = [e; e_delta];

% 3. 前向计算 (算出 u)
net_c1 = Wc1 * xi_c;
Hc = 1 ./ (1 + exp(-net_c1));
u_raw = Wc2 * Hc;

% 限幅
if u_raw > 0.5; u = 0.5;
elseif u_raw < -0.5; u = -0.5;
else; u = u_raw; end

% 4. 反向传播 (更新控制器)
delta3_c = e * J_active; 

dWc2 = xite_c * delta3_c * Hc';
Wc2 = Wc2 + dWc2 + alpha_c * dWc2;

d_sigma_c = Hc .* (1 - Hc) + 0.05;
delta2_c = (Wc2' * delta3_c) .* d_sigma_c;
dWc1 = xite_c * delta2_c * xi_c';
Wc1 = Wc1 + dWc1 + alpha_c * dWc1;

% 5. 更新历史
y_prev = y;

end
\end{lstlisting}

\paragraph{代码详解与工程思考:}
控制器的基本思路大致与符号近似法类似,我在初始化部分同样采用了预训练权值加载的方式,以提升初始控制性能。
但在反向传播阶段引入了从辨识器模块传递过来的 Jacobian 估计值 \texttt{J\_est\_in}。
为了防止极端情况下辨识器输出过小的 Jacobian 导致控制器“失活”,我在代码中设置了一个安全底线:
\begin{lstlisting}
if J_est_in < 0.001
    J_active = 0.001;
else
    J_active = J_est_in;
end
\end{lstlisting} 
这确保了无论辨识器输出如何,控制器始终能够获得一个合理的梯度幅值进行权值更新。

最终神经网络估计法的神经网络直接控制器的仿真结果如下图所示:
\begin{figure}[H]
    \centering
    \includegraphics[width=0.9\textwidth]{figure/direct_control_nnresult.png}
    \caption{神经网络估计法的神经网络直接控制系统仿真结果}
    \label{fig:direct_nn_res}
\end{figure}

\subsubsection{两种方法的性能对比分析}

对比符号近似法,可以明显观察到,神经网络估计法在控制性能上有显著提升,具体体现在以下几个方面:
\begin{itemize}
    \item 超调与振荡减少:波峰波谷处的超调现象明显减小,系统响应更加平滑稳定。
    \item 收敛速度加快:系统达到稳态的时间缩短,控制器能够更快地适应动态变化。
\end{itemize}
不过在跟踪精度方面,符号似然法和神经网络估计法的稳态误差均在$10^{-3}$数量级,跟踪曲线与参考信号几乎重合,表现都很优异,提升效果不大。

综上种种,验证了引入高精度 Jacobian 信息对提升非线性自适应控制性能的关键作用。




\section{任务三:神经网络 PID 控制系统设计}

\subsection{控制系统结构设计}

传统的 PID 控制器参数固定,这就导致当情况不同时,控制器仍在以相同的参数工作,会导致控制效果大打折扣,也就是说,
传统PID难以适应 USV 在复杂海况下的非线性与时变特性。为此,本环节设计了基于 BP 神经网络的自适应 PID 控制系统(BP-PID)。

\subsubsection{系统结构框图}
该控制系统主要由两部分组成:
\begin{enumerate}
    \item 常规 PID 控制器:直接对 USV 进行闭环控制,采用增量式 PID 算法,其参数 $K_p, K_i, K_d$ 由神经网络实时调整。
    \item BP 神经网络:作为一个参数整定器,根据系统的运行状态(误差及其变化率),通过自学习找出当前时刻最优的 PID 参数组合。
\end{enumerate}

系统结构如图所示。

\begin{figure}[H]
    \centering
    \begin{tikzpicture}[auto, node distance=2cm,>=latex',
        block/.style = {draw, rectangle, minimum height=1.2cm, minimum width=2.5cm, align=center, fill=blue!5},
        nnblock/.style = {draw, rectangle, minimum height=1.8cm, minimum width=3.0cm, align=center, fill=green!10, rounded corners},
        sum/.style = {draw, circle, inner sep=2pt, node distance=1.5cm},
        input/.style = {coordinate},
        output/.style = {coordinate}
    ]

        % 1. 定义节点
        \node [input] (input) {};
        \node [sum, right of=input] (sum) {};
        \node [block, right of=sum, node distance=3.5cm] (pid) {\textbf{PID 控制器}\\(增量式)};
        \node [block, right of=pid, node distance=4cm] (plant) {\textbf{被控对象}\\\textbf{(USV)}};
        \node [output, right of=plant, node distance=2.5cm] (out) {};
        
        % BP 神经网络 (放在上方)
        \node [nnblock, above of=pid, node distance=2.5cm] (bp) {\textbf{BP 神经网络}\\(参数自整定)};

        % 2. 信号流连线
        \draw [->, thick] (input) -- node {$y_r$} (sum);
        \draw [->, thick] (sum) -- node {$e$} (pid);
        \draw [->, thick] (pid) -- node [name=u] {$u$} (plant);
        \draw [->, thick] (plant) -- node [name=y] {$y$} (out);
        
        % 反馈
        \draw [->, thick] ($(plant.east)+(1.5cm,0)$) |- (0,-4) -| (sum);
        
        % 3. 神经网络连接
        % 输入到 NN
        \draw [->, dashed] (0.8, 0) |- (bp.west) node [pos=0.2, left] {状态输入 $x$};
        % NN 输出参数到 PID
        \draw [->, thick, blue] (bp.south) -- node [right] {$K_p, K_i, K_d$} (pid.north);
        
        % 4. 误差反向传播 (学习通路)
        \draw [->, dashed, red] (sum.north) |- node [pos=0.7, above] {误差 $e$ (用于梯度计算)} (bp.160);

        % 比较器符号
        \node at (sum.north west) [xshift=-2pt, yshift=2pt] {$+$};
        \node at (sum.south) [yshift=-5pt, xshift=-5pt] {$-$};

    \end{tikzpicture}
    \caption{BP 神经网络自适应 PID 控制系统结构框图}
    \label{fig:bp_pid_structure}
\end{figure}

\subsection{PID 控制算法与权值更新律推导}

\subsubsection{增量式 PID 控制算法}
为了便于计算机实现及与神经网络结合,采用增量式 PID 算法。控制律定义如下:
\begin{equation}
    \begin{aligned}
    u(k) &= u(k-1) + \Delta u(k) \\
         &= u(k-1) + K_p [e(k) - e(k-1)] + K_i e(k) + K_d [e(k) - 2e(k-1) + e(k-2)]
    \end{aligned}
    \label{eq:inc_pid}
\end{equation}
其中,$K_p, K_i, K_d$ 分别为神经网络输出层神经元的输出。

\subsubsection{神经网络权值更新律推导}
定义性能指标函数为 $E(k) = \frac{1}{2} e(k)^2 = \frac{1}{2} (y_r(k) - y(k))^2$。
根据梯度下降法,需调整网络权值 $w$ 以最小化 $E(k)$。

利用链式法则展开权值梯度的偏导数:
\begin{equation}
    \frac{\partial E(k)}{\partial w} = \frac{\partial E(k)}{\partial y(k)} \cdot \frac{\partial y(k)}{\partial u(k)} \cdot \frac{\partial u(k)}{\partial K} \cdot \frac{\partial K}{\partial w}
\end{equation}

各项的具体含义与计算方法如下:

\begin{enumerate}
    \item 误差项:
    \begin{equation}
        \frac{\partial E(k)}{\partial y(k)} = -(y_r(k) - y(k)) = -e(k)
    \end{equation}
    
    \item 对象 Jacobian 信息:
    $\frac{\partial y(k)}{\partial u(k)}$ 反映了被控对象的灵敏度。同 Task 2,此处既可以使用符号近似 $\text{sgn}(\frac{\partial y}{\partial u})$,也可以利用 Task 1 训练好的辨识器模型进行精确估计。
    
    \item PID 梯度项:
    根据公式,控制量 $u(k)$ 对 PID 参数的偏导数为:
    \begin{equation}
        \begin{cases}
            \frac{\partial u(k)}{\partial K_p} = e(k) - e(k-1) \\
            \frac{\partial u(k)}{\partial K_i} = e(k) \\
            \frac{\partial u(k)}{\partial K_d} = e(k) - 2e(k-1) + e(k-2)
        \end{cases}
    \end{equation}
    
    \item 网络内部梯度:
    $\frac{\partial K}{\partial w}$ 取决于神经网络的激活函数及其导数。
\end{enumerate}

在权值更新律中,项 $\frac{\partial K}{\partial w}$ 反映了神经网络内部参数的变化如何传递到输出端(PID 参数)。这一项的计算完全取决于所选用的激活函数及其导数性质。基于代码实现,具体推导如下:
\paragraph{1. 输出层梯度 (Sigmoid 函数与物理缩放)}
输出层神经元采用 Sigmoid 函数将加权和 $net^{(3)}$ 映射到 $(0, 1)$ 区间,并通过增益系数 $S$(即代码中的 \texttt{Scale\_Factors})恢复物理量纲。
输出关系定义为:
\begin{equation}
    K_l = S_l \cdot \sigma(net_l^{(3)}) = S_l \cdot \frac{1}{1 + e^{-net_l^{(3)}}}
\end{equation}
其中 $l=0,1,2$ 分别对应 $K_p, K_i, K_d$。

对权值 $w^{(3)}$ 求偏导,利用 Sigmoid 函数的导数性质 $\sigma'(x) = \sigma(x)(1-\sigma(x))$:
\begin{equation}
    \begin{aligned}
    \frac{\partial K_l}{\partial net_l^{(3)}} &= S_l \cdot \frac{\partial \sigma(net_l^{(3)})}{\partial net_l^{(3)}} \\
    &= S_l \cdot \sigma(net_l^{(3)}) \cdot \left( 1 - \sigma(net_l^{(3)}) \right) \\
    &= \underbrace{K_l}_{\text{当前值}} \cdot \underbrace{\left( 1 - \frac{K_l}{S_l} \right)}_{\text{Sigmoid 导数项}}
    \end{aligned}
\end{equation}

\textbf{代码对应关系:}
该推导严格对应了代码中的输出层梯度计算行:
\begin{lstlisting}
d_g = K_raw .* (1.0 - Sigmoid_Out); 
\end{lstlisting}
这一处理巧妙地利用了函数值本身来计算导数,避免了复杂的指数运算。

\paragraph{2. 隐含层梯度 (Tanh 函数)}
隐含层神经元采用双曲正切函数 (Tanh) 以捕捉误差信号的正负非线性特征。
输出关系定义为:
\begin{equation}
    H_i = \tanh(net_i^{(2)}) = \frac{e^{net_i^{(2)}} - e^{-net_i^{(2)}}}{e^{net_i^{(2)}} + e^{-net_i^{(2)}}}
\end{equation}

其导数具有优良的解析性质 $\tanh'(x) = 1 - \tanh^2(x)$。因此,隐含层输出对输入的灵敏度为:
\begin{equation}
    \frac{\partial H_i}{\partial net_i^{(2)}} = 1 - H_i^2
\end{equation}

\textbf{代码对应关系:}
该推导对应了代码中隐层误差反传的关键一行:
\begin{lstlisting}
d_sigma = (1 - H.^2); % Tanh 导数
\end{lstlisting}

\paragraph{3. 权值更新的链式汇总}
综合上述导数,根据链式法则,完整的权值梯度计算如下:

对于输出层权值 $w_{li}^{(3)}$(连接隐层节点 $i$ 到输出节点 $l$):
\begin{equation}
    \frac{\partial E}{\partial w_{li}^{(3)}} = \delta_l^{(3)} \cdot H_i 
    \quad \text{其中} \quad 
    \delta_l^{(3)} = e(k) \cdot \frac{\partial y}{\partial u} \cdot \frac{\partial u}{\partial K_l} \cdot \underbrace{K_l (1 - \sigma_l)}_{\text{激活函数导数}}
\end{equation}

对于隐层权值 $w_{ij}^{(2)}$(连接输入节点 $j$ 到隐层节点 $i$):
\begin{equation}
    \frac{\partial E}{\partial w_{ij}^{(2)}} = \delta_i^{(2)} \cdot x_j 
    \quad \text{其中} \quad 
    \delta_i^{(2)} = \underbrace{(1 - H_i^2)}_{\text{激活函数导数}} \cdot \sum_{l} \left( \delta_l^{(3)} \cdot w_{li}^{(3)} \right)
\end{equation}

\textbf{最终权值更新律:}
综合上述各项,得到神经网络输出层权值的在线调整公式:
\begin{equation}
    \Delta w(k) = \eta \cdot e(k) \cdot \underbrace{\frac{\partial y}{\partial u}}_{\text{Jacobian}} \cdot \underbrace{\frac{\partial u}{\partial K}}_{\text{PID Input}} \cdot \underbrace{\frac{\partial K}{\partial w}}_{\text{NN Grad}}
\end{equation}
通过该更新律,神经网络能够根据当前的误差趋势,自动增强或减弱 PID 的比例、积分、微分作用,从而实现最优控制。
整体系统结构如图所示。
\begin{figure}[H]
    \centering
    \includegraphics[width=0.9\textwidth]{figure/BP-PID.png} 
    \caption{BP-PID控制系统 Simulink 模型}
    \label{fig:bp_pid_system}
\end{figure}

\subsubsection{系统工作原理}
该控制系统采用了典型的间接自适应控制架构:
\begin{enumerate}
    \item 前向控制通路:BP-PID 控制器根据当前的误差状态 $e(k)$,实时计算 PID 参数 $K_p, K_i, K_d$,并输出控制量 $u(k)$ 驱动 USV。
    \item 并行辨识通路:NN Identifier 与 USV 并联,实时接收输入 $u(k)$ 和输出 $y(k)$。辨识器在线微调权值以逼近系统动态,并计算关键的系统灵敏度信息(Jacobian 信息 $\hat{J} = \frac{\partial \hat{y}}{\partial u}$)。
    \item 信息耦合:辨识器计算出的 Jacobian 信息 $J_{est}$ 被实时传输给 BP-PID 控制器。控制器利用该梯度信息修正权值更新的方向和步长,实现了“基于模型”的参数自整定。
\end{enumerate}

\subsection{核心模块代码实现与分析}

\subsubsection{神经网络辨识器 (NN Identifier)}

我的NN\_Identifier代码与任务二中的在线辨识器相同,故不再赘述。

\subsubsection{BP-PID 控制器 (BP-PID Controller)}
控制器采用 3-5-3 网络结构,核心在于如何利用辨识器传来的 $J_{est}$ 进行权值更新。
其详细代码如下:
\begin{lstlisting}
function [u, K_out] = fcn_BP_PID(yr, y, J_est)

persistent W1 W2 W1_prev W2_prev u_prev y_prev e_prev e_pprev is_init

% --- 1. 超参数设置 ---
xite_k = 0.001; 
alpha  = 0.0; % 动量因子

% --- 2. 初始化 (关键!) ---
if isempty(is_init)
    
    W1 = rand(5, 3) * 0.1; 
    W2 = rand(3, 5) * 0.1;
    
    W1_prev = W1;
    W2_prev = W2;
    
    % 历史状态初始化
    u_prev = 0;
    y_prev = 0;
    e_prev = 0;
    e_pprev = 0;
    
    is_init = 1;
end

% --- 3. 构造 PID 的输入向量 (xc) ---
e = yr - y;                % 当前误差
e_delta = e - e_prev;      % 误差变化
e_ddelta = e - 2*e_prev + e_pprev; % 误差变化的加速度

% 输入向量 (3x1)
xc = [e; e_delta; e_ddelta];

% --- 4. 神经网络前向计算 ---
% >> 隐层计算 (使用 tanh 激活函数)
net1 = W1 * xc;
H = tanh(net1); 

% >> 输出层计算 (使用 Sigmoid 变体,保证 K > 0)
net2 = W2 * H;
% Sigmoid 函数: 把输出压缩到 (0, 1)
Sigmoid_Out = exp(net2) ./ (exp(net2) + 1);


% 分别设置 PID 幅值上限 (消灭高频震荡)
% 定义缩放因子向量 [Kp上限; Ki上限; Kd上限]
Scale_Factors = [50; 0.2; 15]; 

K_raw = Sigmoid_Out .* Scale_Factors;

Kp = K_raw(1);
Ki = K_raw(2);
Kd = K_raw(3);

% --- 5. 执行增量式 PID 控制 ---

du = Kp * (e - e_prev) + Ki * e + Kd * (e - 2*e_prev + e_pprev);

% 算出当前控制量
u_raw = u_prev + du;

% 限幅 (物理约束)
if u_raw > 0.5; u = 0.5;
elseif u_raw < -0.5; u = -0.5;
else; u = u_raw; end

% --- 6. 在线反向传播 (更新权值) ---
% 处理 Jacobian (J_est)
if J_est < 0.001
    J_active = 0.001; % 防止除以0或过小
    sgn_J = 1;        % 默认正方向
else
    J_active = J_est;
    sgn_J = sign(J_est); 
end

% 计算误差梯度项 (delta3)
delta_common = e * J_active; 

% 计算 du 对 K 的偏导数 (对应 PID 三项)
du_dK = [ (e - e_prev);  e;  (e - 2*e_prev + e_pprev) ];

% 输出层梯度 (dK)
d_g = K_raw .* (1.0 - K_raw); % Sigmoid 导数项近似

% 综合梯度 delta3 (3x1向量,对应 Kp, Ki, Kd)
delta3 = delta_common * du_dK .* d_g;

% >> 更新输出层权值 W2
dW2 = xite_k * delta3 * H';
W2_new = W2 + dW2 + alpha * (W2 - W2_prev);

% >> 更新隐层权值 W1
d_sigma = (1 - H.^2);
delta2 = (W2' * delta3) .* d_sigma;

dW1 = xite_k * delta2 * xc';
W1_new = W1 + dW1 + alpha * (W1 - W1_prev);

% 更新权值历史
W2_prev = W2; W2 = W2_new;
W1_prev = W1; W1 = W1_new;

% --- 7. 更新历史状态 ---
u_prev = u;
e_pprev = e_prev;
e_prev = e;
y_prev = y;

% 输出 K 值供观察
K_out = [Kp; Ki; Kd];

end
\end{lstlisting}


\subsubsection{代码核心逻辑详细解析}

\paragraph{1. 输入向量构造与网络初始化}

\begin{lstlisting}
% --- 3. 构造 PID 的输入向量 (xc) ---
e = yr - y;                % 误差
e_delta = e - e_prev;      % 误差变化率
e_ddelta = e - 2*e_prev + e_pprev; % 误差加速度
xc = [e; e_delta; e_ddelta];
\end{lstlisting}
\begin{itemize}
    \item 状态感知:输入层选取了误差 $e$、误差一阶差分 $\Delta e$ 和二阶差分 $\Delta^2 e$。这与 PID 控制律的三个分量(比例、积分、微分)在物理意义上是一一对应的,确保了神经网络能够感知系统的过去、当前和未来。
    \item 初始化策略:我不再在这里使用预处理的操作,而是采用了随机初始化(\texttt{rand})。这是因为 PID 参数的最优值随工况变化较大,且很难预知一个通用的初始 $K$ 值。随机初始化赋予了网络在参数空间内自由搜索的能力。
\end{itemize}

\paragraph{2. 前向传播与参数空间的物理映射}

\begin{lstlisting}
% >> 输出层计算
net2 = W2 * H;
Sigmoid_Out = exp(net2) ./ (exp(net2) + 1); % 归一化到 (0, 1)
% 定义缩放因子向量 [Kp上限; Ki上限; Kd上限]
Scale_Factors = [50; 0.2; 15]; 
K_raw = Sigmoid_Out .* Scale_Factors;
\end{lstlisting}

神经网络的标准输出通常在 $[0, 1]$ 或 $[-1, 1]$ 之间,直接作为 PID 参数显然不符合物理实际。因此,代码引入了 \texttt{Scale\_Factors} 进行物理量纲恢复:
\begin{equation}
    K_{PID} = \text{Scale} \cdot \sigma(net)
\end{equation}
这一设计实现了两个目的:
\begin{enumerate}
    \item 非负约束:利用 Sigmoid 函数的特性,天然保证了 $K_p, K_i, K_d \geq 0$,满足控制稳定性的基本要求。
    \item 搜索空间限制:通过设定上限(如 $K_p \le 50$),防止了网络在初期训练时因权值发散导致参数过大,从而引发系统剧烈震荡。
\end{enumerate}

\paragraph{3. 基于辨识信息的梯度计算}
这是本算法与传统 BP-PID(通常使用符号函数近似 Jacobian)最大的不同点:
\begin{lstlisting}
% 处理 Jacobian (J_est)
if J_est < 0.001
    J_active = 0.001; % 鲁棒性保护
else
    J_active = J_est;
end
% 计算误差梯度项 (delta3)
delta_common = e * J_active; 
\end{lstlisting}

代码引入了辨识器计算出的 $J_{est}$(即 $\frac{\partial y}{\partial u}$)。
不过具体原因已在前文分析,也不再赘述。

\paragraph{4. 权值在线更新律}

\begin{lstlisting}
% 计算 du 对 K 的偏导数 (对应 PID 三项)
du_dK = [ (e - e_prev);  e;  (e - 2*e_prev + e_pprev) ];
% 输出层梯度 (dK)
d_g = K_raw .* (1.0 - K_raw); % 激活函数导数修正
% 综合梯度 delta3
delta3 = delta_common * du_dK .* d_g;
\end{lstlisting}

这段代码严格实现了链式法则 $\frac{\partial E}{\partial w} = \frac{\partial E}{\partial y} \frac{\partial y}{\partial u} \frac{\partial u}{\partial K} \frac{\partial K}{\partial w}$:
\begin{itemize}
    \item \texttt{delta\_common} 对应 $\frac{\partial E}{\partial y} \frac{\partial y}{\partial u} = e \cdot J$。
    \item \texttt{du\_dK} 对应 PID 算法本身对参数的偏导 $\frac{\partial u}{\partial K}$。
    \item \texttt{d\_g} 对应激活函数的导数项 $\frac{\partial K}{\partial net}$。
\end{itemize}
通过这一系列计算,将最终的性能指标误差 $E$ 反向传播回网络权值 $W1, W2$,实现了控制参数的闭环在线优化。

\paragraph{5. 其他工程处理}
关于物理限幅与动量项的处理逻辑,与任务二中采用的方法完全一致,此处不再赘述。


\subsection{仿真结果与性能分析}

首先,PID三个参数的在线变化曲线如下所示:
\begin{figure}[H]
    \centering
    \includegraphics[width=0.9\textwidth]{figure/PID_results.jpg}
    \caption{BP-PID 控制器参数在线调整曲线}
    \label{fig:bp_pid_k_values}
\end{figure}

可以看到,PID 参数在系统运行过程中不断调整,以适应不同的动态需求,后续则因为误差较小而趋于稳定。

系统仿真结果与误差曲线如下所示:
\begin{figure}[H]
    \centering
    \includegraphics[width=0.95\textwidth]{figure/BP_PID_results.png}
    \caption{BP-PID 控制系统仿真结果与误差曲线}
    \label{fig:bp_pid_results}
\end{figure}    

从图像中看出,系统在初始阶段存在一定的超调与振荡,但整体收敛速度较快,稳态误差控制在 $10^{-4}$ 量级,表现出很好的跟踪性能。


该系统成功构建了一个“辨识-控制”相互增强的自适应架构,既保留了 PID 控制结构简单、鲁棒性强的优点,
又赋予了系统处理非线性、时变特性的能力,误差极小,表现出色。



\section{附加题:RBF神经网络自适应控制}

\subsection{问题描述与控制目标}
借鉴模糊自适应控制的设计思想,利用 RBF 神经网络的“万能逼近”特性,设计一种自适应控制器。
目标是针对具有未知非线性动态的 USV 模型,通过在线学习抵消非线性干扰,实现对参考轨迹 $y_r$ 的精确跟踪。

USV 的二阶仿射非线性动力学模型描述为:
\begin{equation}
    \ddot{y} = f(y, \dot{y}) + g \cdot u
\end{equation}
其中,$f(y, \dot{y})$ 为未知的非线性函数(包含水动力阻力等),$g$ 为已知的控制增益。
具体的推论过程我已在上一次模糊控制作业汇报中完成,故这里不再赘述。

\subsection{RBF 神经网络结构设计}
采用径向基函数(RBF)神经网络来逼近未知的 $f(x)$ 。RBF 网络是一种三层前向网络,其映射关系可表示为:
\begin{equation}
    f(x) = W^{*T} h(x) + \varepsilon
\end{equation}
其中:
\begin{itemize}
    \item $x = [\dot{y}]^T$ 为网络输入向量(根据物理特性,阻力主要与速度相关)。
    \item $W^*$ 为理想权值向量。
    \item $h(x) = [h_1, h_2, \dots, h_m]^T$ 为高斯基函数向量,第 $j$ 个节点的输出定义为 [cite: 116]:
    \begin{equation}
        h_j(x) = \exp \left( -\frac{\|x - c_j\|^2}{2b_j^2} \right)
    \end{equation}
    \item $\varepsilon$ 为网络逼近误差。
\end{itemize}

在实际控制中,我使用估计权值 $\hat{W}$ 代替理想权值,则非线性项的估计值为 $\hat{f}(x) = \hat{W}^T h(x)$。

\subsection{控制器设计与稳定性证明}

\subsubsection{控制律设计}
定义跟踪误差 $e = y_r - y$。为了保证系统的渐近稳定性,设计滑模面函数:
\begin{equation}
    s = \dot{e} + \lambda e, \quad \lambda > 0
\end{equation}

对 $s$ 求导,并代入系统方程:
\begin{equation}
    \dot{s} = \ddot{e} + \lambda \dot{e} = (\ddot{y}_r - \ddot{y}) + \lambda \dot{e} = \ddot{y}_r - (f + g u) + \lambda \dot{e}
\end{equation}

为了消除非线性项 $f$ 并使误差收敛,设计包含 RBF 补偿的控制律:
\begin{equation}
    u = \frac{1}{g} \left( -\hat{f}(x) + \ddot{y}_r + K_d \dot{e} + K_p e \right)
\end{equation}
该控制律包含两部分:
1. 非线性补偿项 $-\hat{f}(x)$:利用 RBF 网络抵消未知的模型动态。
2. PD 反馈项 $K_d \dot{e} + K_p e$:保证闭环系统的动态稳定性。

\subsubsection{自适应律推导 (Lyapunov 方法)}
定义 Lyapunov 函数 $V$ 为:
\begin{equation}
    V = \frac{1}{2} s^2 + \frac{1}{2\gamma} \tilde{W}^T \tilde{W}
\end{equation}
其中 $\gamma > 0$ 为自适应学习率,$\tilde{W} = W^* - \hat{W}$ 为权值误差。

对 $V$ 求导,推导可得权值自适应更新律:
\begin{equation}
    \dot{\hat{W}} = -\gamma \cdot s \cdot h(x)
\end{equation}
在离散仿真中,该更新律采用欧拉法实现:$W(k+1) = W(k) - \gamma \cdot s(k) \cdot h(k) \cdot \Delta t$。

整体的仿真模型结构如图所示。
\begin{figure}[H]
    \centering
    \includegraphics[width=0.8\textwidth]{figure/RBF-Adaptive.png}
    \caption{RBF 自适应控制系统 Simulink 模型}
    \label{fig:rbf_adaptive_structure}
\end{figure}

\subsection{核心代码实现与分析}

\subsubsection{核心代码实现}

RBF 自适应控制器的核心代码实现如下:
\begin{lstlisting}
function [u, f_hat, yr, err] = fcn_RBF_Adaptive(t, y, dy)

persistent W c b is_init

% 1. 内置参考信号生成 
A = 0.2;        
w = 0.1;       

yr   = A * sin(w * t);
dyr  = A * w * cos(w * t);
ddyr = -A * w^2 * sin(w * t);

% 2. 参数设置
g = 0.4;       % 物理参数 K/T
gamma = 20.0;  % 自适应学习率 
lambda = 1.0;  % 滑模面参数
Kp = 10.0;     % PD 参数 P
Kd = 5.0;      % PD 参数 D

% RBF 网络结构
node_num = 7; 

if isempty(is_init)
    % 权值初始化
    W = zeros(node_num, 1);
    
    % 中心初始化 (覆盖角速度 dy 的范围 [-1, 1])
    c = linspace(-1, 1, node_num); 
    
    % 宽度初始化
    b = 0.5 * ones(node_num, 1);
    
    is_init = 1;
end

% 3. 计算误差与状态
e = yr - y;
de = dyr - dy;

% 综合误差变量 s
s = de + lambda * e;

% 输出误差供示波器观察 (Export)
err = e;

% 4. RBF 网络前向计算
x_in = dy; % 输入为角速度

h = zeros(node_num, 1);
for j = 1:node_num
    h(j) = exp( - (x_in - c(j))^2 / (2 * b(j)^2) );
end

% 网络输出 (对非线性项的估计)
f_hat = W' * h;

% 5. 计算控制律 u
u_raw = (1/g) * ( -f_hat + ddyr + Kd*de + Kp*e );

% 限幅保护
if u_raw > 0.5; u = 0.5;
elseif u_raw < -0.5; u = -0.5;
else; u = u_raw; end

% 6. 自适应律 

 dW = -gamma * s * h;

% 离散更新
dt = 0.05; 
W = W + dW * dt;

end
\end{lstlisting}


\subsubsection{代码核心逻辑详细解析}

关于输入信号,我原本是在模块外部采用外置的正弦信号发生器生成参考轨迹 $y_r$,但是我在调试过程中发现,
在模块外部引入的过多的delay模块使得波形受影响很大以至于发散,因此我将参考信号的生成内置在了函数内部,
在内部直接计算正弦参考信号及其一阶、二阶导数,以保证参考信号的准确性和实时性,并在模块外加设clock模块以提供时间变量 $t$。

至于自适应学习参数部分,则参考我在模糊自适应调节过程中得到的数据(如g和gamma等),发现后续实验效果较好,故沿用。

\paragraph{1. 内置参考信号与滑模面定义}
\begin{lstlisting}
% --- 3. 计算误差与状态 ---
e = yr - y;
de = dyr - dy;
% 综合误差变量 s (滑模面)
s = de + lambda * e;
\end{lstlisting}

为了导出稳定的自适应律,代码首先定义了滑模变量 $s = \dot{e} + \lambda e$。
\begin{itemize}
    \item 物理意义:$s$ 将跟踪误差 $e$ 及其变化率 $\dot{e}$ 融合为一个标量信号。控制的目标不仅仅是让 $e \to 0$,而是让系统状态被吸引并保持在滑模面 $s=0$ 上。一旦 $s \to 0$,根据微分方程定义,误差 $e$ 必然指数收敛至零。
    \item 在自适应律中的作用:变量 $s$ 后续直接作为神经网络权值更新的驱动信号,保证了权值是沿着“减小滑模面距离”的方向调整的。
\end{itemize}

\paragraph{2. RBF 网络结构与前向计算}
\begin{lstlisting}
% --- 4. RBF 网络前向计算 ---
x_in = dy; % 输入为角速度
h = zeros(node_num, 1);
for j = 1:node_num
    h(j) = exp( - (x_in - c(j))^2 / (2 * b(j)^2) );
end
f_hat = W' * h; % 网络输出 (对非线性项的估计)
\end{lstlisting}

\begin{itemize}
    \item 输入选择 ($x_{in} = dy$):代码选取角速度 $dy$ 作为网络的输入。这是基于物理先验知识:USV 受到的水动力阻尼和非线性干扰主要与运动速度有关。
    \item 高斯基函数:采用径向基函数作为激活函数。通过 \texttt{linspace} 将中心点 $c$ 均匀分布在 $[-1, 1]$ 区间,覆盖了 USV 可能的角速度范围,保证了网络在工作空间内的局部逼近能力。
    \item 零初始权值:代码中 \texttt{W = zeros}。这表明控制器完全不依赖离线训练,具备从“一无所知”开始在线学习并控制系统的能力。
\end{itemize}

\paragraph{3. 基于补偿的控制律设计}
\begin{lstlisting}
% --- 5. 计算控制律 u ---
% u = (1/g) * ( -f_hat + v )
u_raw = (1/g) * ( -f_hat + ddyr + Kd*de + Kp*e );
\end{lstlisting}

该行代码实现了反馈线性化控制律:
\begin{equation}
    u = \frac{1}{g} \left( -\hat{f}(x) + \ddot{y}_r + K_d \dot{e} + K_p e \right)
\end{equation}
\begin{itemize}
    \item 消除非线性:项 $-\hat{f}$(即代码中的 \texttt{-f\_hat})用于抵消系统真实的非线性动态 $f$。如果估计准确($\hat{f} \approx f$),非线性项将被完全消除。
    \item 线性跟踪:剩余项 $\ddot{y}_r + K_d \dot{e} + K_p e$ 构成了一个标准的 PD 跟踪控制器,用于保证线性化后的系统稳定跟踪参考轨迹。
\end{itemize}

\paragraph{4. 权值自适应律 (核心算法)}
\begin{lstlisting}
% --- 6. 自适应律 ---
dW = -gamma * s * h;
% 离散更新
W = W + dW * dt;
\end{lstlisting}

我的权值自适应率直接源于 Lyapunov 函数 $V = \frac{1}{2}s^2 + \frac{1}{2\gamma}\tilde{W}^T\tilde{W}$ 的稳定性分析。
为了保证 $\dot{V} \le 0$,数学推导得出的连续时间更新律为:
\begin{equation}
    \dot{\hat{W}} = -\gamma \cdot s \cdot h(x)
\end{equation}
代码通过 \texttt{dW = -gamma * s * h} 精确实现了这一公式。
\begin{itemize}
    \item 负号:确保能量函数 $V$ 单调递减,即系统是稳定的。
    \item 相关项:权值更新速度取决于滑模误差 $s$(误差越大更新越快)和基函数输出 $h$(只有被激活的神经元才更新权值,体现了 RBF 的局部学习特性)。
\end{itemize}

\subsubsection{f\_hat 分析}
该代码构建了一个严谨的自适应控制闭环。与任务二和任务三的启发式 BP 算法不同,本环节的 RBF 自适应律具有严格的数学稳定性证明。
仿真中 \texttt{f\_hat} 的有效震荡和误差的收敛,完美验证了该算法在处理模型不确定性方面的强大能力。我在此附上我的f\_hat的变化曲线截图如下:
\begin{figure}[H]
    \centering
    \includegraphics[width=0.6\textwidth]{figure/fhat.png}
    \caption{f\_hat变化曲线}
    \label{fig:f_hat}
\end{figure}

非线性逼近能力分析 ($\hat{f}$):
如图所示,$\hat{f}$ 的变化曲线呈现出清晰的物理特征:
\begin{itemize}
    \item 微幅震荡:由于本次仿真中 USV 运动速度较慢(正弦幅值 0.2, 频率 0.1),其受到的实际水动力阻力 $f(y, \dot{y})$ 本身非常微弱。RBF 网络输出的 $\hat{f}$ 保持在 $10^{-1}$ 量级,准确反映了“低速低阻尼”的物理事实,证明网络没有出现过拟合。
    \item 相位同步:$\hat{f}$ 的震荡周期与 USV 的运动周期严格同步,这表明网络成功捕捉到了与速度方向相关的阻尼特性,通过自适应律实时补偿了系统动力学。
\end{itemize}

可以发现,f\_hat成功捕捉到了非线性项的动态变化特征,验证了RBF网络的在线学习能力。


\subsubsection{仿真结果分析}

\begin{figure}[H]
    \centering
    \includegraphics[width=0.9\textwidth]{figure/RBF_results.png}
    \caption{RBF 网络模型仿真结果}
    \label{fig:f_hat_estimation}
\end{figure}

跟踪误差分析:
在 RBF 网络的辅助下,系统的稳态跟踪误差收敛至极小范围,到达了惊人的$10^{-5}$ 量级。RBF 网络负责抵消模型的不确定性,将复杂的非线性系统“线性化”,从而使 PD 控制器能够轻松实现高精度的轨迹跟踪。

\section{心得体会}
本次作业让我深刻体会到神经网络在控制系统中的强大潜力。通过合理设计网络结构和自适应律,神经网络能够在线学习并补偿复杂的非线性动态,大幅提升系统的跟踪性能和鲁棒性。
我自身对于神经网络的学习方面还不够擅长,在本次作业中遇到了不少困难,但通过反复地学习理论知识理解神经网络的根本工作原理,同时借助AI等工具的辅助,最终完成了作业任务,收获颇丰。



\end{document}