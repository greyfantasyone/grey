\documentclass[UTF8]{ctexart}
\usepackage{geometry, CJKutf8}
\geometry{margin=1.5cm, vmargin={0pt,1cm}}
\setlength{\topmargin}{-1cm}
\setlength{\paperheight}{29.7cm}
\setlength{\textheight}{25.3cm}


% useful packages.
\usepackage{amsfonts}
\usepackage{amsmath}
\usepackage{amssymb}
\usepackage{amsthm}
\usepackage{enumerate}
\usepackage{graphicx}
\usepackage{multicol}
\usepackage{fancyhdr}
\usepackage{layout}
\usepackage{listings}
\usepackage{float, caption}

\lstset{
    basicstyle=\ttfamily, basewidth=0.5em
}

% some common command
\newcommand{\dif}{\mathrm{d}}
\newcommand{\avg}[1]{\left\langle #1 \right\rangle}
\newcommand{\difFrac}[2]{\frac{\dif #1}{\dif #2}}
\newcommand{\pdfFrac}[2]{\frac{\partial #1}{\partial #2}}
\newcommand{\OFL}{\mathrm{OFL}}
\newcommand{\UFL}{\mathrm{UFL}}
\newcommand{\fl}{\mathrm{fl}}
\newcommand{\op}{\odot}
\newcommand{\Eabs}{E_{\mathrm{abs}}}
\newcommand{\Erel}{E_{\mathrm{rel}}}

\begin{document}

\pagestyle{fancy}
\fancyhead{}
\lhead{郭子昊, 3230104714}
\chead{数据结构与算法第五次作业}
\rhead{Nov.3rd, 2024}

\section{对remove函数的阐述和分析}
首先,讨论节点t为空的情况,此时没有节点可以删除所以直接返回。
然后,如果x小于当前节点的值,那么递归地在t的左子树中删除x。如果x大于当前节点的元素,则在右子树中删除x。
然后我进行分情况讨论:
如果t的左右子树都不为空,那么通过detachMin函数找到t的右子树中的最小节点minNode,将minNode左子树
指向当前节点的左子树,右子树指向当前节点的右子树,删除当前节点t,同时将t指向minNode。
如果其中一个子树为空,那么通上课代码的思路,保存t到oldNode,将t指向非空的子树,删除oldNode。

关于对程序结果的测试,我创建一个二叉搜索树,插入5,10,15,20,25,30,35,并且打印出树的结构。
然后我删除树最左侧的节点5,再次打印树的结构,发现5被成功删除。
同理我删除有一个子节点的节点30,再次打印树的结构,发现30被成功删除。
然后我删除有两个子节点的节点10,再次打印树的结构,发现10被成功删除。
最后我删除根节点20,再次打印树的结构,发现仍成功删除。
实验程序的结果为:
Initial tree:
5
10
15
20
25
30
35
After removing 5:
10
15
20
25
30
35
After removing 30:
10
15
20
25
35
After removing 10:
15
20
25
35
After removing 20:
15
25
35

\section{测试的结果}

测试结果一切正常。详细的结果见上文的测试程序的设计思路部分。



\end{document}

%%% Local Variables: 
%%% mode: latex
%%% TeX-master: t
%%% End: 
