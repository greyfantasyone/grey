\documentclass[tikz, border=10pt]{standalone}
\usepackage{ctex} % 中文支持
\usepackage{tikz}
% 加载 calc 用于坐标计算,加载 backgrounds 用于调整层级
\usetikzlibrary{shapes.geometric, arrows.meta, positioning, calc, backgrounds}

\begin{document}

\begin{tikzpicture}[
    % 样式定义
    node distance=1.0cm,
    base/.style={
        draw=black!70, 
        thick, 
        align=center, 
        font=\small
    },
    startstop/.style={
        base, 
        rectangle, 
        rounded corners, 
        fill=red!10, 
        minimum width=3cm, 
        minimum height=1cm
    },
    process/.style={
        base, 
        rectangle, 
        fill=orange!10, 
        minimum width=3.5cm, 
        minimum height=1cm
    },
    decision/.style={
        base, 
        diamond, 
        aspect=2.5, 
        fill=blue!10, 
        inner sep=2pt
    },
    stepnode/.style={
        base,
        rectangle,
        fill=yellow!10,
        minimum width=3cm,
        minimum height=0.8cm
    },
    arrow/.style={-Stealth, thick, draw=black!80}
]

    % 1. 流程节点
    \node[startstop] (start) {用户调用 \texttt{game.run()}};
    
    \node[process, below=of start] (turn) {轮到 AI 玩家 \\ \texttt{ai\_player.get\_move()}};
    
    \node[process, below=of turn] (search) {\texttt{MonteCarlo.search()}};
    
    \node[decision, below=of search] (check) {use\_network?};
    
    % 分支
    \node[process, below left=0.8cm and -0.5cm of check] (net) {\texttt{search\_by\_network()}};
    \node[process, below right=0.8cm and -0.5cm of check] (mcts) {\texttt{search\_by\_mcts()}};
    
    % MCTS 循环部分 (手动布局)
    % 先定义循环内的四个步骤
    % 这里的 coordinate 用于辅助定位
    \node[coordinate, below=2.5cm of check] (loop_start) {};
    
    \node[stepnode, below=0.5cm of loop_start] (select) {1. 选择 (Select)};
    \node[stepnode, below=0.3cm of select] (expand) {2. 扩展 (Expand)};
    \node[stepnode, below=0.3cm of expand] (sim) {3. 模拟 (Simulation)};
    \node[stepnode, below=0.3cm of sim] (back) {4. 反向传播 (Backprop)};
    
    % 2. 绘制背景虚线框 (替代之前的 fit)
    % 使用 calc 计算:左上角 = select左上角 + 偏移,右下角 = back右下角 + 偏移
    \begin{scope}[on background layer]
        \draw[dashed, thick, draw=gray!60, fill=gray!5] 
            ($(select.north west)+(-0.3, 0.8)$) rectangle ($(back.south east)+(0.3, -0.3)$);
        
        % 添加虚线框的标题
        \node[above, font=\bfseries\footnotesize] at ($(select.north)+(0, 0.4)$) {build\_montecarlo\_tree() (30秒)};
    \end{scope}

    % 3. 结果
    \node[process, below=1.0cm of back] (best) {返回 \texttt{best\_action}};
    \node[startstop, below=of best] (end) {游戏执行走法};

    % --- 连线 ---
    
    \draw[arrow] (start) -- (turn);
    \draw[arrow] (turn) -- (search);
    \draw[arrow] (search) -- (check);
    
    % 判断连线
    \draw[arrow] (check) -| node[midway, above] {Yes} (net);
    \draw[arrow] (check) -| node[midway, above] {No} (mcts);
    
    % 汇聚到循环 (指向 Select)
    \draw[arrow] (net) |- ($(select.north)+(0,0.2)$) -- (select.north);
    \draw[arrow] (mcts) |- ($(select.north)+(0,0.2)$) -- (select.north);
    
    % 循环内部连线
    \draw[arrow] (select) -- (expand);
    \draw[arrow] (expand) -- (sim);
    \draw[arrow] (sim) -- (back);
    
    % 回环线 (While True)
    \draw[arrow] (back.east) -- ++(0.5,0) |- (select.east);
    % 在回环线旁边添加文字
    \node[right=0.6cm of expand, font=\footnotesize, align=left] {While True \\ (循环)};

    % 结束连线
    % 从虚线框底部穿出,为了美观,从 Back 节点连出,穿过虚线框
    \draw[arrow] (back.south) -- (best);
    \draw[arrow] (best) -- (end);

\end{tikzpicture}

\end{document}